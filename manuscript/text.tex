%Version 3.1 December 2024
% See section 11 of the User Manual for version history
%
%%%%%%%%%%%%%%%%%%%%%%%%%%%%%%%%%%%%%%%%%%%%%%%%%%%%%%%%%%%%%%%%%%%%%%
%%                                                                 %%
%% Please do not use \input{...} to include other tex files.       %%
%% Submit your LaTeX manuscript as one .tex document.              %%
%%                                                                 %%
%% All additional figures and files should be attached             %%
%% separately and not embedded in the \TeX\ document itself.       %%
%%                                                                 %%
%%%%%%%%%%%%%%%%%%%%%%%%%%%%%%%%%%%%%%%%%%%%%%%%%%%%%%%%%%%%%%%%%%%%%

%%\documentclass[referee,sn-basic]{sn-jnl}% referee option is meant for double line spacing

%%=======================================================%%
%% to print line numbers in the margin use lineno option %%
%%=======================================================%%

%%\documentclass[lineno,pdflatex,sn-basic]{sn-jnl}% Basic Springer Nature Reference Style/Chemistry Reference Style

%%=========================================================================================%%
%% the documentclass is set to pdflatex as default. You can delete it if not appropriate.  %%
%%=========================================================================================%%

%%\documentclass[sn-basic]{sn-jnl}% Basic Springer Nature Reference Style/Chemistry Reference Style

%%Note: the following reference styles support Namedate and Numbered referencing. By default the style follows the most common style. To switch between the options you can add or remove �Numbered� in the optional parenthesis. 
%%The option is available for: sn-basic.bst, sn-chicago.bst%  
 
%%\documentclass[pdflatex,sn-nature]{sn-jnl}% Style for submissions to Nature Portfolio journals
%%\documentclass[pdflatex,sn-basic]{sn-jnl}% Basic Springer Nature Reference Style/Chemistry Reference Style
\documentclass[pdflatex,sn-mathphys-num]{sn-jnl}% Math and Physical Sciences Numbered Reference Style
%%\documentclass[pdflatex,sn-mathphys-ay]{sn-jnl}% Math and Physical Sciences Author Year Reference Style



%%\documentclass[pdflatex,sn-aps]{sn-jnl}% American Physical Society (APS) Reference Style
%%\documentclass[pdflatex,sn-vancouver-num]{sn-jnl}% Vancouver Numbered Reference Style
%%\documentclass[pdflatex,sn-vancouver-ay]{sn-jnl}% Vancouver Author Year Reference Style
%%\documentclass[pdflatex,sn-apa]{sn-jnl}% APA Reference Style
%%\documentclass[pdflatex,sn-chicago]{sn-jnl}% Chicago-based Humanities Reference Style

%%%% Standard Packages
%%<additional latex packages if required can be included here>
% 设置中文


\RequirePackage{etex}

\usepackage[UTF8]{ctex}
% \usepackage{xeCJK}
% \setCJKmainfont{Noto Serif CJK SC}
% \usepackage{unicode-math}   % 替代 amssymb + bm + fontspec

\usepackage{subfigure}
\usepackage{tabularx}
\usepackage{graphicx}
\usepackage{float}
\usepackage{epstopdf}
\usepackage{microtype}
\usepackage{enumitem}
\usepackage{multirow}
\usepackage{bbding}
\usepackage{amsmath}
% \usepackage{color}
\usepackage{bm}
\usepackage{cleveref}  
% \usepackage[authoryear,sort&compress,longnamesfirst]{natbib}
\usepackage{xcolor}
\usepackage{hyperref}
\usepackage{amssymb} 
\usepackage{amsthm} 
\usepackage{booktabs}
%%%%
% 可选:让 epstopdf 自动在 \includegraphics 时启用
\epstopdfsetup{outdir=./}
%%%%%=============================================================================%%%%
%%%%  Remarks: This template is provided to aid authors with the preparation
%%%%  of original research articles intended for submission to journals published 
%%%%  by Springer Nature. The guidance has been prepared in partnership with 
%%%%  production teams to conform to Springer Nature technical requirements. 
%%%%  Editorial and presentation requirements differ among journal portfolios and 
%%%%  research disciplines. You may find sections in this template are irrelevant 
%%%%  to your work and are empowered to omit any such section if allowed by the 
%%%%  journal you intend to submit to. The submission guidelines and policies 
%%%%  of the journal take precedence. A detailed User Manual is available in the 
%%%%  template package for technical guidance.
%%%%%=============================================================================%%%%

%% as per the requirement new theorem styles can be included as shown below
\theoremstyle{thmstyleone}%
\newtheorem{theorem}{Theorem}%  meant for continuous numbers
%%\newtheorem{theorem}{Theorem}[section]% meant for sectionwise numbers
%% optional argument [theorem] produces theorem numbering sequence instead of independent numbers for Proposition
\newtheorem{proposition}[theorem]{Proposition}% 
%%\newtheorem{proposition}{Proposition}% to get separate numbers for theorem and proposition etc.
\newtheorem{lemma}{Lemma}[section] 
\newtheorem{assumption}{Assumption}
\theoremstyle{thmstyletwo}%
\newtheorem{example}{Example}%
\newtheorem{remark}{Remark}%

\theoremstyle{thmstylethree}%
\newtheorem{definition}{Definition}%

\raggedbottom
%%\unnumbered% uncomment this for unnumbered level heads

\begin{document}

\title[Article Title]{Predefined Time Prescribed Performance Backstepping Control for Robotic Manipulators with Input Saturation}

%%=============================================================%%
%% GivenName	-> \fnm{Joergen W.}
%% Particle	-> \spfx{van der} -> surname prefix
%% FamilyName	-> \sur{Ploeg}
%% Suffix	-> \sfx{IV}
%% \author*[1,2]{\fnm{Joergen W.} \spfx{van der} \sur{Ploeg} 
%%  \sfx{IV}}\email{iauthor@gmail.com}
%%=============================================================%%

\author[1,2]{\fnm{} \sur{Shuli Liu}}\email{shunnee@163.com}
% \credit{Writing - Original Draft, Methodology, Software, Visualization, Data curation}


\author*[3]{\fnm{} \sur{Yi Liu}}\email{liuyi_hust@163.com}
\author*[3]{\fnm{} \sur{Jingang Liu}}\email{wellbuild@126.com}
% \credit{Resources, Investigation, Formal analysis, Conceptualization, Writing - Review \& Editing}
% \equalcont{These authors contributed equally to this work.}

% \credit{Project administration, Validation, Funding acquisition, Writing - Review \& Editing}
% \equalcont{These authors contributed equally to this work.}
\author*[1,2]{\fnm{} \sur{Yin Yang}}\email{yangyinxtu@xtu.edu.cn}
% \credit{Supervision, Funding acquisition, Writing - Review \& Editing}
% \equalcont{These authors contributed equally to this work.}

\affil[1]{\orgname{School of Mathematics and Computational Science, Xiangtan University}, \orgaddress{\street{Street}, \city{City}, \postcode{411105}, \country{China}}}

\affil[2]{\orgname{National Center for Applied Mathematics in Hunan}, \orgaddress{\street{Street}, \city{City}, \postcode{411105}, \country{China}}}

\affil[3]{\orgname{School of Mechanical Engineering and Mechanics, Xiangtan University}, \orgaddress{\street{Street}, \city{City}, \postcode{411105}, \country{China}}}



% Here goes the abstract
\abstract{
% To simultaneously address the challenges of uncontrollable convergence time and the difficulty of unifying performance constraints in trajectory tracking control for uncertain nonlinear robotic manipulators, this paper proposes a novel adaptive control framework that integrates predefined-time convergence with prescribed performance constraints. Specifically, a new class of Bernstein-polynomial-based performance functions is designed to achieve smooth and adjustable boundaries for both position and velocity errors, significantly reducing dependence on initial error conditions. An exponential barrier Lyapunov function is further introduced to strictly constrain the entire error evolution process. The controller is constructed using the backstepping method and incorporates a radial basis function neural network for online compensation of unknown nonlinear dynamics, thereby enhancing the system’s adaptability and robustness. The proposed approach organically combines predefined-time stability theory with dynamic performance regulation, theoretically guaranteeing the uniform boundedness of all closed-loop signals and prescribed-time convergence of the tracking error. Finally, simulation and experimental verification are carried out on a typical two-degree-of-freedom robotic arm system and a real six-degree-of-freedom collaborative robot platform. The results show that the proposed method outperforms existing schemes in terms of tracking accuracy, convergence speed and robustness, and is particularly suitable for high-performance and time-sensitive robotic systems.
本文研究了具有未知动力学、有界扰动以及执行器饱和条件下的机器人机械臂轨迹跟踪控制问题,提出了一种立于任何初始状态条件的预定义时间预设性能约束的自适应反步控制方法。针对传统预设性能控制中存在的初值奇异性和误差变换非光滑问题,针对每个通道独立设计了结合多项式性能函数与误差缩放函数的预定义时间误差变换结构,确保系统状态在预定义时间内严格且平滑地满足全局预设性能约束。
为处理反步控制及执行器饱和输入引起的计算复杂度爆炸问题,设计了一种预定义时间饱和补偿器,以同时消除命令滤波与输入饱和的不利影响。此外,引入一阶滑模扰动观测器实时估计并补偿系统中未知的有界扰动,利用径向基函数神经网络对系统未建模动态进行辨识和补偿,从而提高了控制系统的鲁棒性和跟踪精度。同时,结合预定义时间理论和自适应动态障碍李雅普诺夫函数,严格证明了闭环系统在预定义时间内的全局稳定性和误差收敛性。
数值和实验结果结果验证了所提出方法的有效性,与现有方法相比,本文方法在响应速度、稳态精度及鲁棒性能方面表现出明显优势。

}
\keywords{Prescribed performance control, predefined-time stability, adaptive dynamic barrier Lyapunov function, backstepping control, input saturation, robotic manipulators' trajectory tracking.
}
\maketitle

% Main text

下面是在你原有排版基础上**逐句打磨、修正符号与措辞**后的版本(保留结构与编号;只做必要的更正与增润)。

---

\subsection{System description}



求$V_1$ 的时间导数为
V_{j,i}
\begin{equation}\label{eq:25}
	\begin{aligned}
\dot V_1
&=\sum_{i=1}^{n}\left[
\frac{\rho_{1,i}^4\,z_{1,i}}{\big(\rho_{1,i}^{2}-z_{1,i}^{2}\big)^{2}}\dot z_{1,i}
-
\frac{\rho_{1,i}\,z_{1,i}^{4}}{\big(\rho_{1,i}^{2}-z_{1,i}^{2}\big)^{2}}\dot \rho_{1,i}
\right]\\
% &=\sum_{i=1}^{n}\left[
% \frac{\rho_{1,i}^4\,z_{1,i}}{\big(\rho_{1,i}^{2}-z_{1,i}^{2}\big)^{2}}(z_{2,i}-\dot q_{d,i}+\alpha^{f}_i+\zeta_i)-
% \frac{\rho_{1,i}\,z_{1,i}^{4}}{\big(\rho_{1,i}^{2}-z_{1,i}^{2}\big)^{2}}\dot \rho_{1,i}
% \right]
\end{aligned}
\end{equation}

记
$
P_{j,i}=\frac{\rho_{j,i}^4}{(\rho_{j,i}^2-z_{j,i}^2)^2}>0,
Q_{j,i}=\frac{ z_{j,i}^4}{(\rho_{j,i}^2-z_{j,i}^2)^2}>0,
\Phi_{j,i}=\frac{z_{j,i}\left(\rho_{j,i}^{2}-z_{j,i}^{2}\right)}{\rho_{j,i}^{4}}=\frac{z_{j,i}(\rho_{j,i}^2-z_{j,i}^2)}{P_{j,i}}, j\in\{1,2\}
$,
$\dot z_{1,i}= z_{2,i}-\dot q_{d,i}+\alpha^{f}_i+\zeta_i$, 将它们代入 BLF 导数后,并根据公式(),得到
$$
\dot V_1=\sum_{i=1}^n\!\Big[P_{1,i} z_{1,i}z_{2,i}+P_{1,i}z_{1,i}(\alpha_i+\zeta_i-\dot q_{d,i})-P_{1,i}z_{1,i}\tilde\alpha_i-Q_{1,i}\rho_{1,i} \dot\rho_{1,i}\Big]
$$

Let ${V_{j,i}}=\frac{1}{2}\frac{\rho_{j,i}^2 z_{j,i}^2}{\rho_{j,i}^2-z_{j,i}^2}, \Psi(V_{j,i})=\frac{\pi}{ \eta T_p}\Big((V_{j,i})^{\,1-\eta/2}+n^{-\frac{\eta}{2} }(V_{j,i})^{\,1+\eta/2}\Big)$, 我们设计
$$
\mathcal{K}_{1,i}(z_{j,i},\rho_{j,i})
=\frac{\rho_{j,i}^2}{2}\frac{\Psi(V_{j,i})}{V_{j,i}}
=\frac{\pi\,\rho_{j,i}^2}{2\eta T_p}\Big((V_{j,i})^{-\eta/2}+n^{-\frac{\eta}{2} }(V_{j,i})^{\eta/2}\Big).
$$

把它代回,可设计虚拟控制律为:
\begin{equation}\label{eq:25}
\alpha_i = \dot q_{d,i}-\zeta_i+\frac{z_{1,i}^3}{\rho_{1,i}^3}\,\dot\rho_{1,i}
-\mathcal{K}_{1,i}(z_{1,i},\rho_{1,i})\Phi_{1,i}
-k_{1,i} \rho_{1,i}^2 \Phi_{1,i},
\end{equation}
where $k_1=\mathrm{diag}\{k_{1,1},k_{1,i},\cdots,k_{1,n}\}>0$, $k_{1,i}>0$,

则一步 Lyapunov 导数化简为

$$
\dot V_1
\le \sum_{i=1}^n \Big[ P_{1,i}z_{1,i} z_{2,i}
-\Psi(V_{1,i})-P_{1,i}z_{1,i}\tilde\alpha_i- k_{1,i}\frac{\rho_{1,i}^2 z_{1,i}^2}{\rho_{1,i}^2-z_{1,i}^2} 
\Big].
$$


In the second step, a BLF is used as an energy function for the error dynamics. We choose:
\begin{equation}\label{eq:25}
	V_2= \frac{1}{2}\sum_{i=1}^{n} \frac{\rho_{2,i}^2 z_{2,i}^2}{\rho_{2,i}^2-z_{2,i}^2}. 
\end{equation}


Using $\dot z_{2,i}=\dot x_{2,i}-\dot\alpha_i^{f}-\dot\zeta_i$, we obtain the time derivative of $V_2$ is as
\begin{equation}
	\begin{aligned}
		\dot V_2& = \sum_{i=1}^n \Big[\,P_{2,i}z_{2,i}\,(\dot x_{2,i}-\dot\alpha_i^{f}-\dot\zeta_i)-Q_{2,i}\rho_{2,i}\,\dot\rho_{2,i}\,\Big]\\
		&
		 = \sum_{i=1}^n \left[\,P_{2,i}z_{2,i}\,\left(f_i(x)+h_i(x,t)+g_i(u_i-\Delta u_i)+d'(t)-\dot\alpha_i^{f}-\delta_i\ +\ \mu_2\,\zeta_i\ +\ g_i\,\Delta u_i-\tfrac{z_{2}^{3}}{\rho_{2,i}^{3}}\dot\rho_{2,i} \right)\right]
	\end{aligned}
\end{equation}

Let $Q_2=\mathrm{col}\{Q_{2,i}\}$,
$P_1=\mathrm{col}\{P_{1,i}\}$,
$\Phi_2=\mathrm{col}\{\Phi_{2,i}\}$. The control input torque is designed as
\begin{equation}\label{eq:tau-cmd}
\begin{aligned}
u =&C(q,\dot{q})x_2 + G(q)\\
&+M(q)\left[\dot\alpha^{f}+\delta -\mu_2\,\zeta-\frac{P_1}{P_2}z_1+\tfrac{z_{2}^{3}}{\rho_{2,i}^{3}}\dot\rho_{2,i}-\mathcal{K}_{2}(z_{2},\rho_{2,i}) \Phi_{2}
-\hat{h}(\chi)
-\hat \omega
-k_s\,\mathrm{sgn}(z_{2})-k_{2}\rho_{2,i}^2 \Phi_{2,i}\right]
\end{aligned}
\end{equation}


where $k_2=\mathrm{diag}\{k_{2,1},k_{2,i},\cdots,k_{2,n}\}>0$, $k_{2,i}>0$, $k_s>0$.

Let $\chi=\{q,\dot{q}\}\in\mathbb{R}^m$ be the regressor and $\psi(\chi)\in\mathbb{R}^{N}$ the normalized basis. Approximate the structured uncertainty by 
\( \hat h_i(\chi)=\hat \theta_i^\top \psi_i(\chi)\),
with $\theta_i\in\mathbb{R}^{N}  $ adapted online by
\begin{equation}\label{eq:theta-law}
\dot{\hat{\theta}}_i = \varrho_i\,P_{2,i} z_{2,i}\psi(\chi) - \kappa_i\,\hat{\theta}_i,\qquad
\varrho_i,\kappa_i>0.
\end{equation}


将结构不确定估计误差和外扰的剩余项定义为
$\omega(x,t)=h(x)-\hat h(\chi)+ M^{-1}(q) d(t)$。
为避免直接微分速度,采用一阶跟踪微分器作为观测器
\begin{equation}
	\begin{cases}
\dot\vartheta =-\omega_d\,\vartheta+\omega_d\,x_2,  \\
\hat{\dot x}_2=\omega_d(x_2-\vartheta),   \\
\dot{\hat \omega}= -\Lambda_o\,\hat \omega
+\Lambda_o\left(\hat{\dot x}_2 - f(x) - M^{-1}(q)\,\mathrm{sat}(u) - \hat h(\chi)\right),
\end{cases}
\end{equation}
where $\Lambda_o=\mathrm{diag}\{\lambda_{o,i}\}>0, \omega_d>0, \vartheta(0)=x_2(0)$,
设观测误差为 $\tilde \omega=\omega-\hat\omega$.
根据公式可得到
\begin{equation}
	\begin{aligned}
	\dot V_2
\le&+\sum_{i=1}^n \left( - k_{2,i}\frac{\rho_{2,i}^2 z_{2,i}^2}{\rho_{2,i}^2-z_{2,i}^2}-\Psi(V_{2,i})-P_{2,i}z_{2,i}\tilde{\omega}_i  \right) -k_s\sum_{i=1}^n P_{2,i} \left\lvert z_{2,i}\right\rvert +P_{2,i}z_{2,i}\left(h(x)-\hat h(\chi)\right)
\end{aligned}
\end{equation}



\begin{equation}\label{eq:45}
	\begin{aligned}
		\dot{V} \le&\sum_{i=1}^n \left( - k_{1,i}\frac{\rho_{1,i}^2 z_{1,i}^2}{\rho_{1,i}^2-z_{1,i}^2}-\Psi(V_{1,i})-P_{1,i}z_{1,i}\tilde\alpha_i\right)
		+\sum_{i=1}^n \left( - k_{2,i}\frac{\rho_{2,i}^2 z_{2,i}^2}{\rho_{2,i}^2-z_{2,i}^2}-\Psi(V_{2,i})-P_{2,i}z_{2,i}\tilde{\omega}_i  \right)\\
		& -k_s\sum_{i=1}^n P_{2,i} \left\lvert z_{2,i}\right\rvert +P_{2,i}z_{2,i}\left(h(x)-\hat h(\chi)\right)+\tilde\alpha^\top \dot{\tilde \alpha}+\tfrac{1}{\gamma_o}\tilde \omega^\top \dot{\tilde \omega}+\sum_{i=1}^n\tfrac{1}{\varrho_i}\tilde \theta_i^\top \dot{\tilde \theta}_i \\
		\le&\sum_{i=1}^n \left( - k_{1,i}\frac{\rho_{1,i}^2 z_{1,i}^2}{\rho_{1,i}^2-z_{1,i}^2}-\Psi(V_{1,i})\right)
		+\sum_{i=1}^n \left( - k_{2,i}\frac{\rho_{2,i}^2 z_{2,i}^2}{\rho_{2,i}^2-z_{2,i}^2}-\Psi(V_{2,i})  \right)\\
		& -P_{2,i}z_{2,i}\tilde{\omega}_i +\tfrac{1}{\gamma_o}\tilde \omega^\top \dot{\tilde \omega}-k_s\sum_{i=1}^n P_{2,i} \left\lvert z_{2,i}\right\rvert+\tilde\alpha^\top \dot{\tilde \alpha}-P_{1,i}z_{1,i}\tilde\alpha_i+P_{2,i}z_{2,i}\left(h(x)-\hat h(\chi)\right)+\sum_{i=1}^n\tfrac{1}{\varrho_i}\tilde \theta_i^\top \dot{\tilde \theta}_i 
	\end{aligned}
\end{equation}

$$
\rho_{2,i}^2 - z_{2,i}^2 \le \rho_{2,i}^2
$$

\begin{equation}\label{eq:45}
	\begin{aligned}


	\end{aligned}
\end{equation}

% \bibliographystyle{sn-mathphys-num}
% \bibliography{sn-bibliography}

% \input{sn-article.bbl}
\section*{Statements and Declarations}
\subsection*{Funding}

This work is supported by the National Key Research and Development Program of China (2023YFC3008802), the National Natural Science Foundation of China (52075465), and the Key Products of Hunan Province's Manufacturing Industry "Unveiled and Leading" Project (2023GXGG018), the Project of Scientific Research Fund of the Hunan Provincial Science and Technology Department (2023JJ50021, 2023GK2029, No.2024JC1003, No.2024JJ1008, 2024JJ2051, 2023GK2026), "Algorithmic Research on Mathematical Common Fundamentals" Program for Science and Technology Innovative Research Team in Higher Educational Institutions of Hunan Province of China, the Hunan Provincial Department of Education Excellent Youth Program (23B0162).

\subsection*{Conflicts of Interest}
The authors declare no conflict of interest.

\subsection*{Author contribution}
Shuli Liu: Writing-Original Draft, Methodology, Software, Visualization, Data curation. Yi Liu:Resources, Investigation, Formal analysis, Conceptualization, Writing-Review \& Editing. 
Jingang Liu: Project administration, Validation, Funding acquisition, Writing-Review \& Editing. 
Yin Yang: Supervision, Funding acquisition, Writing-Review \& Editing. 
All authors have reviewed and approved the final version of the manuscript.
\subsection*{Data Availability}
The data that support the findings of this study are available from the corresponding author upon reasonable request.

\end{document}


